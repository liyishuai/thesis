\section{Specifying and Testing Protocols}
Modelling languages for specifying protocols can be partitioned into three
styles, according to \textcite{anand2013orchestrated}: (1) {\em Process-oriented}
notations that describe the SUT's behavior in a procedural style, using various
domain-specific languages like our interaction trees; (2) {\em State-oriented}
notations that specify what behavior the SUT should exhibit in a given state,
which includes variants of labelled transition systems (LTS); and (3) {\em
  Scenario-oriented} notations that describe the expected behavior from an
outside observer's point of view ({\it i.e.,} ``god's-eye view'').

The area of model-based testing is well-studied, diverse, and difficult to
navigate~\cite{anand2013orchestrated}.  Here we focus on techniques that have
been practiced in testing real-world programs, which includes notations (1) and
(2).  Notation (3) is infeasible for protocols with nontrivial nondeterminism,
because the specification needs to define observer-side knowledge of the SUT's
all possible internal states, making it complex to implement and hard to reason
about, as shown in \autoref{fig:etag-tester}.

Language of Temporal Ordering Specification (LOTOS)~\cite{Bolognesi1987} is the
ISO standard for specifying OSI protocols.  It defines distributed concurrent
systems as {\em processes} that interact via {\em channels}, and represents
internal nondeterminism as choices among processes.

Using a formal language strongly insired by LOTOS, \textcite{torxakis-dropbox}
implemented a test generation tool for symbolic transition systems called
TorXakis, which has been used for testing Dropbox~\cite{torxakis-dropbox}.

TorXakis provides limited support for internal nondeterminism.  Unlike our
testing framework that incorporates symbolic evalutation, TorXakis enumerates
all possible values of internally generated data, until finding a corresponding
case that matches the tester's observation.  This requires the server model to
generate data within a reasonably small range, and thus cannot handle generic
choices like HTTP entity tags, which can be arbitrary strings.

\textcite{netsem} have developed rigorous specifications for transport-layer
protocols TCP, UDP, and the Sockets API, and validated the specifications
against mainstream implementations in FreeBSD, Linux, and WinXP.  Their
specification represents internal nondeterminism as symbolic states of the
model, which is then evaluated using a special-purpose symbolic model checker.
They focused on developing a post-hoc specification that matches existing
systems, and wrote a separate tool for generating test cases.

\section{Reasoning about Network Delays}
For property-based testing against distributed applications like Dropbox,
\textcite{testing-dropbox} have introduced ``conjectured events'' to represent
uploading and downloading events that nodes may perform at any time invisibly.

\textcite{pkt-dyn} symbolised the time elapsed to transmit packets from one end
to another, and developed a symbolic-execution-based tester that found
transmission-related bugs in Linux TFTP upon certain network delays.  Their
tester used a fixed trace of packets to interact with the server, and the
generated test cases were the packets' delay time.
