This chapter explores methodologies in specifying and automatically testing
interactive systems.
\autoref{sec:related-validate} compares different specification
techniques for testing purposes.
\autoref{sec:related-harness} then discusses practices in developing test
harnesses.

\section{From Specifications to Validators}
\label{sec:related-validate}

Different testing scenarios exhibit various challenges that motivate the
specification design.  This section partitions the validation techniques by the
languages used for developing the specifications.

\subsection{State machine specification: Quviq QuickCheck}
Property-based testing with QuickCheck has been well adopted in industrial
contexts~\cite{Hughes2016}.  The specification is a boolean function over traces
{\it i.e.} the validator.  My solutions to addressing internal and external
nondeterminism are inspired by practices in QuickCheck.

\paragraph{Internal nondeterminism}
My \http specification was initially written as a QuickCheck property.  Before
handling preconditions like \inlinec{If-Match} and \inlinec{If-None-Match}, the
validator implements a deterministic server model and compares its behavior with
the observations, as shown in the \ilc{validate} function in
\autoref{sec:interactive-testing}.  When expanding the specification to cover
conditional requests, I implemented the ad hoc validator by manually translating
the trace into tester-side knowledge, as shown in \autoref{fig:etag-tester}.

The complexity in describing ``what behavior is valid'' motivated me to rephrase
the specification.  I applied the idea of model-based
testing~\cite{broy2005model}, and specified the protocol in terms of ``how to
produce valid behavior''.  My specification represented internal nondeterminism
with symbolic variables.  The validator then checks whether the trace is
producible by the symbolic specification, by reducing the producibility problem
to constraint solving.

\paragraph{External nondeterminism}
\citet{testing-dropbox} have used QuickCheck to test Dropbox.  The specification
does not involve internal nondeterminism, but does handle external
nondeterminism that local nodes may communicate with the server at any time.
This is done by introducing ``conjectured events'' to represent the possible
communications.  The validator checks if the conjectured events can be inserted
to somewhere in the trace to make it producible by the model.

To specify servers' allowed observations delayed by the network, \citet{cpp19}
introduced the concept of ``network refinement''.  The network may scramble the
traces by delaying some messages after others, with one exception: If the client
has received response \ilc A before it sends request \ilc Q, then by causality,
the server-side trace must have sent response \ilc A before receiving request
\ilc Q.  Upon observing a client-side trace \ilc T, our QuickCheck validator
searches for a server-side trace that (i) can be reasonably scrambled into trace
\ilc T, and (ii) complies with the protocol specification.

My network model design in \autoref{sec:external-nondet} was inspired by these
ideas of conjecturing the environment's behavior.  Instead of inserting
conjectured communication events or reordering the trace, I specified the
network as an independent module and composed it with the server specification.
This provides more flexibility in specifying asynchronous systems: (i) To adjust
the network configurations, e.g., limiting the buffer size or the number of
concurrent connections, we only need to adjust the network model without
modifying the server's events.  (ii) Instead of carefully defining the space of
valid scrambled traces for each network configuration, we can derive it from the
network model by dualization.  (iii) The network model is reusable and allows
specifying various protocols on top of it.

\subsection{Process algebra: LOTOS and TorXakis}
Language of Termporal Ordering Specification (LOTOS)~\cite{lotos} is the ISO
standard for specifying OSI protocols.  It defines distributed concurrent
systems as {\em processes} that interact via {\em channels}.  Using a formal
language strongly inspired by LOTOS, \citet{torxakis-dropbox} implemented a test
generation tool called TorXakis, and used it for testing Dropbox.

TorXakis supports internal nondeterminism by defining a process for each
possible value.  This requires the space of invisible values to be finite.  In
comparison, I represented invisible values as symbolic variables and employed
constraint solving that can handle inifitite space of data like strings and
integers.

As for external nondeterminism, TorXakis hardcodes all channels between each
pair of processes, assuming no new process joins the system.  Whereas in my
network model, ``channels'' are the ``source'' and ``destination'' fields of
network packets, which allows specifying a server that exposes its port to
infinitely many clients.

\subsection{Transition systems: NetSem and Modbat}
Using labelled transition systems (LTS), \citet{netsem} have developed rigorous
specification for TCP, UDP, and the Sockets API.  To handle internal
nondeterminism in real-world implementations, they symbolized the model states,
which is then evaluated with a special-purpose model symbolic model checker.
The tester helped revealing their post-hoc specifications' mismatch with
mainstream systems like FreeBSD, Linux, and WinXP.  I borrowed the idea of
symbolic evaluation in validating observations, and used it for detecting
mainstream servers' violations against the formalized RFC specification.

\citet{modbat} have generated test cases for Java network API, which involves
blocking and non-blocking communications.  Their abstract model was based on
extended finite state machines (EFSM), and could capture bugs in the network
library \verb|java.nio|.  Their validator rejects the SUT upon unexpected
exceptions or observations that fail its {\em encoded} assertions.  In
comparison, assertions in my validator are {\em derived} from the abstract
model, which covers full functional correctness of the SUT modulo external
nondeterminism.

\section{Test Harnesses}
\label{sec:related-harness}
This section explores techniques of generating and shrinking test inputs.
\autoref{sec:related-gen} compares different heuristics used by test generators;
\autoref{sec:related-shrink} explains existing shrinking techniques for
interactive testing scenarios.

\subsection{Generator Heuristics}
\label{sec:related-gen}

In addition to state-based and trace-based heuristics discussed in
\autoref{sec:heuristics}, other kinds of heuristics can be applied to generating
inputs for various testing scenarios.

\paragraph{Constraint-based heuristics}
The reason for introducing heuristics is to increase the chance of triggering
invalid behavior.  I specified the heuristics as ``how to produce interesting
input'', while in some cases, it's more convenient to specify ``what inputs are
interesting''.  For example, well-typed lambda expressions can be easily defined
in terms of typing rules, but are less intuitive to enumerate by a generator.

Narrowing~\cite{narrowing} allows generating data that satisfy certain
constraints.  \citet{gengood} have applied the idea of narrowing to the
QuickChick testing framework in Coq~\cite{quickchick}, representing constraints
as inductive relations.  The relations are compiled into efficient generators
that produce satisfying data.

The narrow-based generator in QuickChick has been used during my preliminary
experiment with \http, where I defined a well-formed relation for HTTP requests
to guide the generator.  However, the QuickChick implementation was unstable and
failed to derive generators as my type definition becomes more and more complex.
This constraint-based heuristics may be integrated to my current testing
framework by interpreting QuickChick's \ilc{Gen} (generator) monad into
the \ilc{IO} monad used by the test harness, provided the generator derivation
failure gets resolved.

\paragraph{Coverage-based heuristics}
Another strategy to increase the chance of revealing invalid behavior is to
cover more execution paths of the SUT.  This idea is applied to fuzz
testing~\cite{fuzz}, with popular implementations AFL~\cite{afl} and
Honggfuzz~\cite{honggfuzz}, and combined with property-based testing by
\citet{fuzzchick}.

Coverage-based testers mutate the test inputs and observe the programs'
execution paths.  An input is considered interesting if it causes the program to
traverse a previously unvisited path.  The interesting inputs will be mutated
and rerun to cover potentially more paths.

To track the program's execution paths, coverage-based heuristics need to
instrument the SUT during compilation, making it inapplicable for black-box
testing.  When fuzzing interactive systems like web servers, the SUT is run in a
simulator where the requests are provided as files instead of network packets.
The responses are ignored by the comtemporary fuzzing tools, which only checks
whether the SUT crashes or hangs and does not care about functional correctness
like the responses' contents.  To produce a trace for validation, the SUT needs
to be carefully modified to record its send and receive events.  Addressing
these challenges would enable coverage-based heuristics to be integrated in
interactive testing of systems' functional correctness.

\subsection{Shrinking Interactive Tests}
\label{sec:related-shrink}

To address inter-execution nondeterminism in Quviq QuickCheck,
\citet{Hughes2016} introduced the idea of shrinking abstract representations of
test input.  The tester first generates a {\em script} using state-based
heuristics, and instantiates the script with the tester's runtime state.  The
scripts can be shrunk and adapted to new runtimes when rerunning the test.

The language design of my test harness is inspired by the QuickCheck approach,
and extends it in two dimensions:
\begin{enumerate}
\item The QuickCheck framework assumes synchronous interactions, where the
  requests are function calls that immediately return.  When testing
  asynchronous systems {\it e.g.} networked servers, the responses might be
  indefinitely delayed by the environment, which would block the QuickCheck
  tester's state transition.
  
  To generate test inputs asynchronously, I implemented a non-blocking algorithm
  for instantiating scripts into requests.  When a dependant observation is
  absent due to delays or loss by the environment, the test harness tries to
  instantiate the request using other observations, instead of waiting for the
  observation from the environment.

\item QuickCheck requires the test developer to specify a runtime state to guide
  the generator, and define the state transition rules for each interaction.
  The heuristics are implemented based on the tester's point of view.

  In comparison, my framework replaced these requirements with state- and
  trace-based heuristics, where the model state was exposed by the underlying
  specification, and the trace was automatically recorded by the test harness.
  This allows test developers to design heuristics based on the implementation's
  point of view, instead of reasoning on the implementation's internal and
  external nondeterminism based on its observations.
\end{enumerate}
