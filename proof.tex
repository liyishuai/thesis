So far I have introduced the $\Prog$ language for writing specifications and
shown how to construct a validator from it.  This section shows how to prove the
soundness and completeness of all validators dualized from $\Prog$-based
specifications:
\[\begin{array}{r@{\;}l}
\forall p:\Prog,&\letin{s}{\serverOf(p)}\\
&\letin{v}{\validatorOf(p)}\\
&\rejSound v s\wedge\rejComplete v s\\
&\ie\text{, }\forall t:\List(Q\times A),\\
&\qquad\valid s t\iff\accepts v t\\
&\qquad\ie\text{, }\exists s',\behaves s t s'\iff\exists v',\behaves v t v'
\end{array}\]

To apply the proof techniques in \autoref{sec:correctness}, I design the loop
invariant in \autoref{sec:proof-invariant}.  I then use the invariant to prove
the hypotheses for soundness and completeness in \autoref{sec:proof-sound}
and \autoref{sec:proof-complete}, respectively.  The entire proof is formalized
in the Coq proof assistant.

\subsection{Designing the loop invariant}
\label{sec:proof-invariant}
Let $\beta=\Set((\Nat\to\Var)\times\Set\constraint)$ be the validator state type
and $\sigma=\Nat\to\Int$ the server state type.  Then the loop invariant
$\Reflects{(v:\beta)}{(s:\sigma)}$ is a binary relation between the validator
state $v$ and the server state $s$.  To define this relation, I first discuss
the semantics of validator states.

Each validation state in the validator state consists of an address-variable
mapping and a set of constraints over the variables.  The validation state
defines a space of server states.  A validator accepts a trace if it has a
validation state whose constraints are satisfiable, which indicates the
existence of a server that produces the trace.  A corresponding server state can
be constructed from any assignment that satisfies the validation state's
constraints:
\[\vs^{\asgn}\quad\triangleq\quad \mathit{addr}\mapsto \asgn!(\vs!\mathit{addr})\]
Let $(\vs:\Nat\to\Var)$ be a mapping in the validation state,
$(\asgn:\Var\to\Int)$ be an assignment of variables, then
$(\vs^\asgn:\Nat\to\Int)$ is a server state that maps each address
$!\mathit{addr}$ to the value assigned by $\asgn$ to the address' corresponding
variable in $\vs$.

\begin{definition}[Loop invariant for $\Prog$-based validators]
Validator state $v$ reflects server state $s$ (written ``$\Reflects vs$'') if
the server state lies within the space of servers defined by some validation
state $(\vs,cs)\in v$.  That is, there exists an assignment that relates the
server state and the validation state:
\begin{align*}
\Reflects vs\quad\triangleq\quad&\exists(\vs,\cs)\in
  v,\;\exists\asgn,\\
  &\satisfy{\asgn} \cs\;\wedge\;\vs^{\asgn}=s
  \end{align*}
\end{definition}

Having defined the loop invariant, we only need to instantiate the hypotheses
in \autoref{sec:qac-soundness}--\ref{sec:qac-completeness} with $\Prog$-based
definitions.  The rest of this section proves the hypotheses for establishing
validators' soundness and completeness.

\subsection{Proving rejection soundness}
\label{sec:proof-sound}
\begin{lemma}[\ref{eq:rs1}]
\[\begin{array}{ll}
\text{If:}&
\vs=(\_\mapsto\#0)\qquad
\cs=\{\#0=0\}\qquad
s=(\_\mapsto0)\\
\text{Then:}&\Reflects{\{(\vs,\cs)\}}{s}
\end{array}\]
\end{lemma}
\begin{proof}
Since $(\vs,\cs)$ is the only element in the validator state, we only need to show
that:
\[\exists(\asgn:\Var\to\Int),\quad\satisfy{\asgn} \cs\;\wedge\;\vs^{\asgn}=s\]

By constructing the assignment as: \(\asgn=(\_\mapsto0)\)

We have: \(\#0^{\asgn}=0\)

Thus: \(\satisfy{\asgn} \cs\)

We also know that: \[\forall k, \quad\asgn!(\vs!k)=0=(s!k)\]

Therefore: \(\vs^{\asgn}=s\)
\end{proof}

\begin{lemma}[\ref{eq:rs2}]
  \begin{align*}
    &\forall(q,c,a:\Int)(s,s':\sigma)(v:\beta),\\
    &\triggers sc{(q,a)}s'\;\wedge\;\Reflects{v}{s}\\
    &\implies\exists v':\beta,\;\behaves v{(q,a)}v'\;\wedge\;\Reflects{v'}{s'}
  \end{align*}
\begin{proof}
The invariant $\Reflects{v}{s}$ tells us that $v$ contains a pre-validation
state that reflects the pre-step server state $s$:
\[\exists(\vs,\cs)\in v,\;\exists asgn,\quad\satisfy{\asgn} \cs\;\wedge\;{\vs}^{\asgn}=s\]

The $\vstep'_p$ function in \autoref{fig:dualize} leads to a set of
post-validation states.  We need to show that some state in it reflects the
post-step server state $s'$:
\[\exists(\vs',\cs')\in\vstep_p'(q,a)(\vs,\cs),\;\exists\asgn',\quad\satisfy{\asgn'}\cs'\;\wedge\;{\vs'}^{~\asgn'}=s'\]

The server's internal choice $c$ was provided, so we know the server's entire
execution path.  By induction on the $\Prog$ syntax, we can choose the right
post-validation state $(\vs',\cs')$ by computing each branch condition, and
deduce the satisfying assignment $\asgn'$ by computing each write operation's
source value and destination variable.
\end{proof}
\end{lemma}

\subsection{Proving rejection completeness}
\label{sec:proof-complete}

\begin{lemma}[\ref{eq:rc1}]
\begin{align*}
\forall(q,a:\Int)(v,v':\beta),\;&\behaves v{(q,a)}v'\\
&\implies\exists s':\sigma,\;\Reflects{v'}{s'}
\end{align*}
\begin{proof}
Since the $\vstep_p$ function in \autoref{fig:dualize} checks the nonemptiness
of the result, we know that $v'$ must be nonempty.  Consider validation state
$(\vs',\cs')\in v'$.  Since $\vstep'_p$ checks that $(\solvable \cs')$, we know
that:
\[\exists \asgn,\quad\satisfy{\asgn}cs'\]

Let:
\(s'=\vs'^{~\asgn}\)

Then we have:
\begin{align*}
&(\vs',\cs')\in v'\quad\wedge\quad\satisfy{\asgn}{\cs'}\quad\wedge\quad\vs'^{~\asgn}=s'\\
&\ie\text{, }\Reflects{v'}{s'}\tag*{\qedhere}
\end{align*}
\end{proof}
\end{lemma}

\begin{lemma}[\ref{eq:rc2}]
\begin{align*}
&\forall(q,a:\Int)(v,v':\beta)(s':\sigma),\\
&\behaves v{(q,a)}v'\;\wedge\;\Reflects{v'}{s'}\\
&\implies\exists(s:\sigma)(c:\Int),\;\triggers sc{(q,a)}s'\;\wedge\;\Reflects{v}{s}
\end{align*}
\begin{proof}
We first construct the pre-step server state $(s:\sigma\mid\Reflects{v}{s})$.
We then compute the internal choice $c$ and prove the server step $\triggers
sc{(q,a)}s'$.

The definition of $\Reflects{v'}{s'}$ says:
\[\exists (\vs',\cs')\in v',\;\exists \asgn,\quad \satisfy{\asgn}{\cs'}\;\wedge\;\vs'^{~\asgn}=s'\]

From the definition of $\vstep_p$, we know that:
\[\exists (\vs,\cs)\in v,\quad (\vs',\cs')\in\vstep'_p(q,a)(\vs,\cs)\]

Since $\vstep'_p$ monotonically increases the set of constraints, we have
$\cs\subseteq \cs'$.  Therefore: \[\satisfy{\asgn}{\cs}\]

Let: \(s=\vs^{\asgn}\)

Then we have:
\begin{align*}
&(\vs,\cs)\in v\quad\wedge\quad\satisfy{\asgn}{\cs}\quad\wedge\quad \vs^{\asgn}=s\\
&\ie\text{, }\Reflects{v}{s}
\end{align*}

From the definition of $\vstep'_p$, the validator first creates a fresh variable
to represent the server's internal choice, so we can deduce the internal choice
from the assignment:
\[x_c=\Fresh(\vs,\cs)\qquad c=\asgn!x_c\]

Now we need to show that the server's loop body $\sstep_p(q,c)(s)$ results in
response $a$ and post-execution state $s'$.  Since the post-validation state
$v'$ simulates $s'$ and guarantees the response to be $a$, we only need to prove
the post-execution state to be $s'$.

We have known that:
\[s=\vs^\asgn\qquad s'=\vs'^{~\asgn}\qquad\behaves{(\vs,\cs)}{(q,a)}{(\vs',\cs')}\]

Therefore, we can prove $\triggers sc{(q,a)}s'$ by induction on the $\Prog$
syntax, showing that every write or branch operation preserves $\asgn$'s ability
to unify the server state with the validation state.  This leads the
post-execution state to be unifiable against $\vs'$ using $\asgn$, thus to be
$s'$.
\end{proof}

The core of this proof is to find the internal choice $c$ for server step
$\triggers sc{(q,a)}s'$.  The choice was computed with the assignment deduced
from the loop invariant.  The assignment contains the value of all symbolic
variables, which includes all choices made by the server, past and future.  The
loop invariant requires the assignment to explain the past choices, but cannot
predict future choices.  Therefore, we can only infer the server step from the
validator step using backward induction.
\end{lemma}

\begin{lemma}[\ref{eq:rc3}]
\[\begin{array}{ll}
\text{If:}&
\vs=(\_\mapsto\#0)\qquad
\cs=\{\#0\equiv0\}\qquad
s_0=(\_\mapsto0)\\
\text{Then:}&\{s\mid\Reflects{\{(\vs,\cs)\}}{s}\}=\{s_0\}
\end{array}\]
\begin{proof}
The requirement for $s$ says:
\[\exists \asgn:\Var\to\Int,\quad\satisfy{\asgn}{\cs}\quad\wedge\quad \vs^{\asgn}=s\]

The constraint satisfaction tells us that:
\(\asgn!0=0\)

We then have:
\[\forall k:\Nat,\quad s!k=\asgn!(vs!k)=\asgn!0=0=s_0!k\]

Therefore, $s_0$ is the only server state that $(\vs,\cs)$ simulates.
\end{proof}
\end{lemma}

Now we have proven that all $\Prog$-based validators satisfy the hypotheses
defined in \autoref{sec:qac-soundness}--\ref{sec:qac-completeness}, and we can
conclude that the validators constructed by dualization are sound and complete.
Next I'll show how to apply this dualization technique to test real-world
programs.
