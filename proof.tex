So far I have introduced the $\Prog$ language for writing specifications and
shown how to construct a validator from it.  This section shows how to prove the
soundness and completeness of all validators dualized from $\Prog$-based
specifications:
\[\begin{array}{r@{\;}l}
\forall p:\Prog,&\letin{s}{\serverOf(p)}\\
&\letin{v}{\validatorOf(p)}\\
&\rejSound v s\wedge\rejComplete v s\\
&\text{\it i.e. }\forall t:\List(Q\times A),\\
&\qquad\valid s t\iff\accepts v t\\
&\qquad\text{\it i.e. }\exists s',\behaves s t s'\iff\exists v',\behaves v t v'
\end{array}\]

To apply the proof techniques in \autoref{sec:correctness}, I design the loop
invariant in \autoref{sec:proof-invariant}.  I then use the invariant to prove
the hypotheses for soundness and completeness in \autoref{sec:proof-sound}
and \autoref{sec:proof-complete}, respectively.

\subsection{Loop invariant}
\label{sec:proof-invariant}
Let $\beta=\Set((\Nat\to\Var)\times\Set\constraint)$ be the validator state
type, and $\sigma=\Nat\to\Int$ be the server state type, then the loop invariant
$\Reflects{(v:\beta)}{(s:\sigma)}$ is a binary relation between the validator
state $v$ and the server state $s$.

A validator state consists of an address-variable mapping and a set of
constraints over the variables.

\subsection{Proving rejection soundness}
\label{sec:proof-sound}

\subsection{Proving rejection completeness}
\label{sec:proof-complete}

\lys{Under construction:}

The hypotheses in the proof strategy are based some loop invariant, which
depends on the modelling language.  We need to define the invariant for the
language, and show that it is preserved between the server and validator steps.

A $\Prog$-based validator maintains a set of validation states, each state
corresponds to a possible execution path of the server model.

A validation state is accepting if its constraints are satisfiable, {\it i.e.}
there exists an assignment of the symbolic variables that can unify the trace
with the server model.

The validator accepts the trace if any of its validation states is accepting,
which indicates that some execution path of the server model can produce the
trace.

Given an accepting validation state, we can construct the server steps that
produce the trace, using the assignment $(\Var\to\Int)$ that satisfies the
constraints.  This assignment evaluates internal choices' symbolic variables
into concrete values, and evaluates the validator's key-variable mapping
$(\Nat\to\Var)$ to the server's key-value mapping $(\Nat\to\Int)$.

Therefore, we only need to show that each server and validator step preserves
the existence of such assignment that relates their states, thus defines the
invariant:

\begin{definition}[Loop invariant between $\Prog$-based specification and validator]
  Validator state $v$ simulates server state $s$ if it contains a validation
  state $(vs,cs)$ that {\em reflects} the server state, {\it i.e.}  (1) There
  exists an assignment $asgn$ that can satisfy the constraints $cs$; and (2) The
  key-variable mapping $vs$ can be evaluated with $asgn$ (written as
  $vs^{asgn}$) into a key-value mapping that is equivalent with
  $s$:\footnote{For the rest of this section,
  $\beta=\Set((\Nat\to\Var)\times\Set\constraint)$ represents the validator
  state type, and $\sigma=\Nat\to\Int$ represents the server state type.}
\[\begin{array}{lll} \Reflects{(v:\beta)}{(s:\sigma)}&\triangleq& \exists((vs,cs)\in
  v)(asgn:\Var\to\Int),\satisfy{asgn} cs\wedge vs^{asgn}\equiv s\\
  vs^{asgn}&\triangleq&addr\mapsto asgn!(vs!addr) \end{array}\]
\end{definition}

\paragraph{Applying proof strategy}
Having defined the loop invariants, we only need to instantiate the QAC-generic
proof strategy with $\Prog$-based definitions.  If the hypotheses are all
satisfied, then we have the soundness and completeness guarantee of every
validator derived from $\Prog$ models.

\begin{lemma}[\ref{eq:rs1}]
\[\begin{array}{ll}
\text{If:}&
vs=(\_\mapsto\#0)\qquad
cs=\{\#0\equiv0\}\qquad
s=(\_\mapsto0)\\
\text{Then:}&\Reflects{\{(vs,cs)\}}{s}
\end{array}\]
\end{lemma}
\begin{proof}
Since $(vs,cs)$ is the only element in the validator state, we only need to show
that:
\[\exists(asgn:\Var\to\Int),\satisfy{asgn} cs\wedge vs^{asgn}\equiv s\]

By constructing the assignment as: \[asgn=(\_\mapsto0)\]

We have: \[\#0^{asgn}=0\]

Thus: \[\satisfy{asgn} cs\]

We also know that: \[\forall k, asgn!(vs!k)=0=(s!k)\]

Thus: \[vs^{asgn}\equiv s\qedhere\]
\end{proof}

\begin{lemma}[\ref{eq:rs2}]
  \begin{align*}
    &\forall(p:\Prog)(q,c,a:\Int)(s,s':\sigma)(v:\beta),\\
    &\sstep_p(q,c,s)=(a,s')\wedge\Reflects{v}{s}\\
    &\implies\exists v':\beta,\vstep_p(q,a,v)=\Some{v'}\wedge\Reflects{v'}{s'}
  \end{align*}
\begin{proof}
The invariant $\Reflects{v}{s}$ tells us that $v$ contains a validation state
that reflects the server state $s_0$:
\[\exists(vs,cs)\in v,\exists asgn:\Var\to\Int,\quad\satisfy{asgn} cs\wedge {vs}^{asgn}\equiv s\]

Since the server's internal choice was provided, we can compute the server's
actual execution path.  For each small step of the server's execution, we can
construct its corresponding validator small step, based on the derivation rules
in \autoref{sec:dualization}.  By making the same internal choice and branch
decisions as the server did, we can construct the assignment that unifies the
validator with the server.  The proof details are shown in \autoref{sec:rs2-proof}.
\end{proof}
\end{lemma}

\begin{lemma}[\ref{eq:rc1}]
\begin{align*}
\forall(p:\Prog)(q,a:\Int)(v,v':\beta),\;&\vstep_p(q,a,v)=\Some{v'}\\
&\implies\exists s':\sigma,\Reflects{v'}{s'}
\end{align*}
\begin{proof}
Since $\vstep_p$ checks the nonemptiness of the result, we know that $v'$ must
be nonempty.  Consider validation state $(vs',cs')\in v'$.  Since $\vstep'_p$
checks that $(\solvable cs')$, we know that:
\[\exists asgn,\quad\satisfy{asgn}cs'\]

Let:
\[s'=vs'^{~asgn}\]

Then we have:
\begin{align*}
&(vs',cs')\in v'\wedge \satisfy{asgn}{cs'}\wedge vs'^{~asgn}\equiv s'\\
&\textit{i.e. }\Reflects{v'}{s'}
\end{align*}
\end{proof}
\end{lemma}

\begin{lemma}[\ref{eq:rc2}]
\begin{align*}
&\forall(p:\Prog)(q,a:\Int)(v,v':\beta)(s':\sigma),\\
&\vstep_p(q,a,v)=\Some{v'}\wedge\Reflects{v'}{s'}\\
&\implies\exists(s:\sigma)(c:\Int),\sstep_p(q,c,s)=(a,s')\wedge\Reflects{v}{s}
\end{align*}
\begin{proof}
We first construct the initial server state $(s:\sigma\mid\Reflects{v}{s})$.  We
then compute the internal choice $c$ and construct the server step that
corresponds with the validator step.

The definition of $\Reflects{v'}{s'}$ says:
\[\exists (vs',cs')\in v',\exists asgn,\quad \satisfy{asgn}{cs'}\wedge vs'^{~asgn}\equiv s'\]

From the definition of $\vstep_p$, we know that:
\[\exists (vs,cs)\in v,\quad \vstep'_p(q,a,(vs,cs))=(vs',cs')\]

Since $\vstep'_p$ monotonically increases set of constraints, we have
$cs\subseteq cs'$.  Therefore: \[\satisfy{asgn}{cs}\]

Let: \[s=vs^{asgn}\]

Then we have:
\begin{align*}
&(vs,cs)\in v\wedge\satisfy{asgn}{cs}\wedge vs^{asgn}\equiv s\\
&\textit{i.e. }\Reflects{v}{s}
\end{align*}

From the definition of $\vstep'_p$, the validator first creates a fresh variable
to represent the server's internal choice.  Let:
\[x_c=\Fresh(vs,cs)\qquad c=asgn!x_c\]

We now have a server step $\sstep_p(q,c,s)$, and need to show that it results in
response $a$ and post-execution state $s'$.  Since the post-validation state
$v'$ simulates $s'$ and guarantees the response to be $a$, we only need to show
that the server step is reflected in the validator.  This is done by analyzing
the server's execution path, proving that each derivation rule preserves such
small-step reflection.  The proof details are shown in \autoref{sec:rc2-proof}
\end{proof}
\end{lemma}

\begin{lemma}[\ref{eq:rc3}]
\[\begin{array}{ll}
\text{If:}&
vs=(\_\mapsto\#0)\qquad
cs=\{\#0\equiv0\}\qquad
s_0=(\_\mapsto0)\\
\text{Then:}&\{s\mid\Reflects{\{(vs,cs)\}}{s}\}=\{s_0\}
\end{array}\]
\begin{proof}
The requirement for $s$ says:
\[\exists asgn:\Var\to\Int,\quad\satisfy{asgn}{cs}\wedge vs^{asgn}=s\]

The constraint satisfaction tells us that:
\[asgn!0=0\]

We then have:
\[\forall k:\Nat,\quad s!k=asgn!(vs!k)=asgn!0=0=s_0!k\]

Therefore, $s_0$ is the only server state that $(vs,cs)$ simulates.
\end{proof}
\end{lemma}

Now we have proven that all $\Prog$-based validators satisfy the hypotheses
defined in \autoref{sec:strategy}, and conclude that these validatos are sound
and complete.  The entire proof is formalized in the Coq proof assistant.

The main idea of the proof is to show the reflection between the server and the
validator, by constructing the assignments that unifies them.  This also answers
why proving rejection completeness requires backward induction: The assignment
evaluates the symbolic variables during the validation process, which includes
all choces made by the server, past and future.  An assignment might include
wrong predictions about the server's future choices, in which case the validator
will drop it upon contradicting observations.  By the end of validation, the
surviving assignment can let us reconstruct a server's execution path, by
infering its internal choices.

So far I have presented the theory of constructing validators with correctness
guarantee.  Next I'll explain how to apply this theory to test real-world
programs.
