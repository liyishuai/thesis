This section implements heuristics for generating test inputs.  I'll use the
HTTP tester as an example to show how to make requests more interesting, by
parameterizing them over the model state (\autoref{sec:heuristic-state}) and the
trace (\autoref{sec:heuristic-trace}).

\subsection{State-based heuristics}
\label{sec:heuristic-state}
The model state may instruct the test generator to produce more interesting test
inputs.  For example, consider the \ilc{random_path} generator in
\autoref{line:random-path} of \autoref{fig:naive-generator}.  One way to improve
it is to generate more paths that have corresponding resources on the server:
\begin{coq}
  Definition gen_path (state: list (path * resource)) : IO path :=
    let paths: list path := map fst state in
    freq [(90, oneof paths);
          (10, random_path)].
\end{coq}

Here I modify the server model's state type \ilc{sigma} from \ilc{(path ->
  resource)} in \autoref{fig:if-match-server} into \ilc{(list (path *
  resource))}, which has the same expressiveness but allows the generator to
access the list of all \ilc{paths} in the server state.  The generator chooses
from these existent paths in 90\% of the cases, as assigned by the \ilc{freq}
combinator.  The remaining 10\% are still generated randomly, to discover how
the SUT handles nonexistent paths.

For the \ilc{gen_packet} generator in \autoref{fig:naive-generator}, replacing
its \ilc{random_path} with the improved \ilc{gen_path} would generate more
interesting request targets.  This requires the \ilc{gen_packet} to carry the
server state to instantiate \ilc{gen_path}.

As shown in \autoref{fig:backtrack}, the \ilc{GenPacket} generator is triggered
when the tester wants to observe a packet from itself to the SUT.
\ilc{fig:symbolic-observer} then shows that such \ilc{FromObserver} expectation
happens when the symbolic model \ilc{Emit}s a packet.  Such \ilc{Emit} event
only happens when the server wants to receive a packet in
\autoref{fig:net-compose}.  The \ilc{Recv} events are triggered by the server
model in \autoref{fig:if-match-server}, which iterates over the server state
\ilc{sigma}.

Therefore, I extend the server's \ilc{Recv} event type to include the server
state:
\begin{coq}
  Variant qaE: Type -> Type :=
    Recv : sigma      -> qaE packet
  | Send : packet -> qaE unit.
\end{coq}

Now when the server wants to receive a request, it triggers \ilc{(Recv state)},
where \ilc{(state: sigma)} contains the server's paths and resources at that
point.  The \ilc{state} argument is then carried to the generator, by adding
parameters to the event types along the interpretation:
\begin{coq}
  Variant netE: Type -> Type :=
    Emit  : packet -> netE unit
  | Absorb: sigma      -> netE packet.

  Variant observeE : Type -> Type :=
    FromObserver   : sigma -> observeE concrete_packet
  | ToObserver     : observeE concrete_packet.

  Variant genE: Type -> Type :=
    GenPacket : sigma -> genE concrete_packet
  | GenBool   : genE bool.

  Definition gen_packet: sigma -> IO concrete_packet.
\end{coq}

As a result, when instantiating the \ilc{(GenPacket state)} event in
\autoref{fig:execute}, we can feed the \ilc{gen_packet} function with argument
\ilc{state}, so that \ilc{gen_path} can generate interesting paths based on the
server state.

\subsection{Trace-based heuristics}
\label{sec:heuristic-trace}

When the SUT makes internal choices {\it e.g.} generating ETags, the
specification represents them as symbolic variables.  These variables' concrete
value are not stored in the specification state, but may be observed during
execution.  For example, when an HTTP server responds to a GET request, it might
include the resource's ETag as shown in \autoref{sec:internal-nondeterminism}.
