The Deep Specification project~\cite{deepspec} aims at building a web server and
guarantee its functional correctness with respect to formal specification of the
network protocol.

It is possible, but tricky, to write an ad hoc tester for \http by manually
``dualizing'' the behaviors described by the informal specification documents
(RFCs).  The protocol document describes {\em how} a valid server should handle
requests, while the tester needs to determine {\em what} responses received from
the server are valid.  For example, ``If the server has revealed some resource's
ETag as \inlinec{"foo"}, then it must not reject requests targetting this
resource conditioned over \inlinec{If-Match: "foo"}, until the resource has been
modified''; and ``Had the server previously rejected an \inlinec{If-Match}
request, it must reject the same request until its target has been modified.''
\autoref{fig:etag-tester} shows a hand-written tester for checking this bit of
ETag functionality; we hope the reader will agree that this testing logic is not
straightforward to derive from the informal ``server's eye'' specifications.

Networked systems are naturally concurrent, as a server can be connected with
multiple clients.  The network might delay packets indefinitely, so messages
sent via different channels may be reordered during transmission.  When the
tester observes messages sent and received on the client side, it should allow
all observations that can be explained by the combination of a valid server + a
reasonable network environment between the server and clients.
