In this section, I introduce the ``query-answer-choice'' (QAC) language family
for writing specifications and validators for network protocols that involve
internal nondeterminism.  Languages in the QAC family can specify protocols of
various message types, server states, and internal choices, by instantiating the
specification language with different type arguments.

\subsection{Specifying protocols with server models}
\label{sec:qac-model}
Network protocols can be specified with ``reference implementations'', \ie,
model programs that exhibit the space of valid behaviors.  For client-server
systems such as WWW, we can specify networked servers as programs that receive
queries and compute the responses.  Here I model the server programs with a data
structure called state monad.

\begin{definition}[State monad]
  Let $S$ be the state type, and $A$ be the result type.  Then type $(S\to
  A\times S)$ represents a computation that, given a pre state, yields a result
  and the post state.  This computation is pronounced a ``state monad with state
  type $S$ and result type $A$''.

  For example, let the state be a key-value mapping $(K\to V)$, then we can
  define \ilc{get} and \ilc{put} computations as follows:
  \begin{align}
    \tag{1}&\mathtt{get}:K\to((K\to V)\to V\times(K\to V))\\
    \tag{2}&\mathtt{get}(k)(f)\triangleq (f(k), f)\\
    \tag{3}&\mathtt{put}:K\times V\to((K\to V)\to ()\times(K\to V))\\
    \tag{4}&\mathtt{put}(k,v)(f)\triangleq((),\update f k v)
  \end{align}

  These function definitions should be read as:
  \begin{enumerate}
    \item The \ilc{get} function takes a key as argument, and constructs a
      state monad with state type $(K\to V)$ and result type $V$.
    \item Given argument $k$ of type $K$, $\mathtt{get}(k)$ takes a mapping $f$
      as pre state and yields the mapped value $f(k)$ as result.  The post state
      is the original mapping $f$ unchanged.
    \item The \ilc{put} function takes a key-value pair as argument, and
      constructs a state monad with state type $(K\to V)$ and result type
      ``$()$'' (unit type, which corresponds to \inlinec{void} return type in
      C/Java functions).
    \item Given argument $(k,v)$ of type $(K\times V)$, $\mathtt{put}(k,v)$
      takes a mapping $f$ as pre state and substitues its value at key $k$ with
      $v$.  The post state is the substituted mapping $\update f k v$.
  \end{enumerate}
\end{definition}

Now we can define the server model in terms of state monad:

\begin{definition}[Deterministic server model]
  \label{def:qaserver}
  A deterministic server is an infinite loop whose loop body takes a query and
  produces a response.  The server definition consists of the loop body and a
  current state:
  \[\mathsf{DeterministicServer}\triangleq\sigT{S}{(Q \to S \to A \times S) \times S}\]
  This type definition is pronounced as: A deterministic server has an initial
  state of some type $S$.  Its loop body takes a request of type $Q$ and
  computes a state monad with state type $S$ and result type $A$, where type $A$
  represents the response.

  Note that the server's state type is existentially quantified~\cite{tapl},
  while its query and response types are not.  This is because a protocol
  specification only defines the space of valid traces, and doesn't require the
  implementation's internal state to be a specific type.

  An instance of server model is written as:
  \[\existT S \sigma (\sstep,state_0)\]
  This expression is pronounced as: The server state is of type $\sigma$.  Its
  loop body is function $\sstep$ (which has type $Q\to\sigma\to A\times\sigma$)
  and its initial state is $state_0$ (which has type $\sigma$).
\end{definition}

For example, consider a compare-and-set (CMP-SET) protocol: The server stores a
number \inlinec n.  If the client sends a request that is smaller than \inlinec
S, then the server responds with \inlinec 0.  Otherwise, the server sets
\inlinec n to the request and responds with \inlinec 1:
\begin{lstlisting}[style=customc]
  int n = 0;
  while (true) {
    int request = recv();
    if (request <= n) send(0);
    else { n = request; send(1); }
  }
\end{lstlisting}

Such a server can be modelled as:
\begin{align*}
  \existT{S}{\Int}{(&\lam{(q)(n)}{\begin{cases}
        (0,s)&q\le n\\
        (1,q)&\mathrm{otherwise}
    \end{cases}}\\
    ,&\;0)}
\end{align*}
In general, servers' responses and transitions might depend on choices that are
invisible to the testers, so called internal nondeterminism, as discussed in
\autoref{sec:internal-nondeterminism}.  I represent the space of nondeterminstic
behaviors by parameterizing it over the server's internal choice.

\begin{definition}[Nondeterministic server model]
  \label{def:server}
  A nondeterministic server is an infinite loop whose loop body takes a query
  and an internal choice to produce a response.  The nondeterministic server
  definition extends \autoref{def:qaserver} with a choice argument of type $C$:
  \[\Server\triangleq\sigT S{(Q\times C\to S\to A\times S)\times S}\]
\end{definition}

Consider changing the aforementioned CMP-SET into compare-and-reset (CMP-RST):
When the request is greater than \inlinec S, the server may reset \inlinec S to
any arbitrary number:
\begin{lstlisting}[style=customc]
  int arbitrary();
  int n = 0;
  while (true) {
    int request = recv();
    if (request <= n) send(0);
    else { n = arbitrary(); send(1); }
  }
\end{lstlisting}
Its corresponding server model can be written as:
\begin{align*}
  \existT{S}{\Int}{(&\lam{(q,c)(n)}{
      \begin{cases}
        (0,n)& q\le n\\
        (1,c)&\mathrm{otherwise}
      \end{cases}
    }\\
    ,&\;0)}
\end{align*}
This model represents the space of uncertain behavior with the internal choice
parameter of type integer.  For any value $(c:\Int)$, the server is allowed to
reset $S$ to $c$.

\subsection{Valid traces of a server model}
By specifying protocols with server models, we can now instantiate the trace
validity notation ``$\valid s t$'' in \autoref{def:compliance} in terms of
operational semantics.

\begin{definition}[Server transitions]
  \label{def:server-step}
  Upon request $q$ and choice $c$, server model $s$ can step to $s'$ yielding
  response $a$ (written ``$\triggers sc{(q,a)}s'$~'') if and only if the
  response and the post model can be computed by the $\stepServer$ function:
\begin{align*}
  &\triggers sc{(q,a)}s'\quad\triangleq\quad\stepServer(q,c)(s)=(a,s')\\
  &\stepServer:Q\times C \to \Server \to A\times \Server \\
  &\stepServer(q,c)(s)\triangleq\\
  &\qquad\letin{(\existT{S}{\sigma}{(\sstep, state)})}{s}\\
  &\qquad\letin{(a,state')}{\sstep(q,c)(state)} \\
  &\qquad(a, \existT{S}{\sigma}{(\sstep, state')})
\end{align*}
The $\stepServer$ function takes a query and a choice and computes a state monad
with state type $\Server$ and result type $A$, by pattern matching on argument
$(s:\Server)$.  Let $\sigma$ be the server state type of $s$, $\sstep$ the loop
body, and $(state:\sigma)$ the current state of $s$.  Then $\stepServer(q,c)(s)$
computes the result $(a:A)$ and the post state $(state':\sigma)$ using the
$\sstep$ function, and substitutes the server's pre-step $state$ with the
post-step $state'$.
\end{definition}

\begin{definition}[Trace validity in QAC]
  \label{def:trace-validity}
  In the QAC language family, a trace is a sequence of $Q\times A$ pairs.  When
  specifying a protocol with a $\Server$ model, a trace $t$ is valid per
  specification $s$ if and only if it can be {\em produced} by the server model:
  \[\valid s t\quad\triangleq\quad\exists s',\behaves s t s'\]
  Here the producibility relation in \autoref{sec:concepts} is expanded with an
  argument $s'$ representing the post-transition state, pronounced
  ``specification $s$ can produce trace $t$ and step to specification $s'$~'':
  \begin{enumerate}
  \item A server model can produce an empty trace and step to itself:
    \[\behaves s \nil s\]
  \item A server model can produce a non-empty trace if it can produce the head
    of the trace and step to some server model that produces the tail of the
    trace:
    \[\behaves s {t+(q,a)} s_2\quad\triangleq\quad\exists s_1,c,\behaves s t s_1\wedge\triggers {s_1}c{(q,a)}s_2\]
  \end{enumerate}
\end{definition}

\subsection{Validating traces}
The validator takes a trace and determines whether it is valid per the protocol
specification.

\begin{definition}[Validator]
A validator is an infinite loop whose loop body takes a pair of query and
response and determines whether it is valid or not.  The validator iterates over
a state of some type $V$.  Given a $Q\times A$ pair, the loop body may return a
next validator state or return nothing, written as type ``$\option V$'':
\[\begin{array}{lll}
  \Validator&\triangleq&\sigT{V}{(Q\times A\to V\to\option V)\times V}\\
  \option X&\triangleq&\Some(x:X)\mid\None
\end{array}\]
\end{definition}

For example, a validator for the CMP-SET protocol is written as:
\begin{align*}
  \existT{V}{\Int}{(&\lam{(q,a)(v)}{
      \begin{cases}
        \If a\Is 1\Then\Some v\Else\None&q\le v\\
        \If a\Is 1\Then\Some q\Else\None&\mathrm{otherwise}
      \end{cases}
    },\\
    &0)}
\end{align*}
Here the validator state is the same as the server model's.  The loop body computes
the expected response and compares it with the observed response.  If they are
the same, then the next server state is used as the next validator state.
Otherwise, the function returns $\None$, indicating that the response is
invalid.

Having defined the validator type, we can now instantiate the trace acceptation
notation ``$\accepts v t$'' in \autoref{def:tester} in terms of operational
semantics.

\begin{definition}[Validator transitions]
  \label{def:validator-step}
  Validator $v$ can consume request $q$ and response $a$ and step to $v'$
  (written ``$\behaves v{(q,a)}v'$~'') if and only if the post validator can be
  computed by the $\stepValidator$ function:
\begin{align*}
  &\behaves v{(q,a)}v'\quad\triangleq\quad\stepValidator(q,a)(v)=\Some v'\\
  &\stepValidator:Q\times A\to\Validator\to\option\Validator\\
  &\stepValidator(q,a)(\existT{V}{\beta}{(\vstep,state)})\\
  &\qquad\triangleq\begin{cases}
  \Some{(\existT{V}{\beta}{(\vstep,state')})} & \vstep(q,a,state)=\Some{state'} \\
  \None & \vstep(q,a,state)=\None
  \end{cases}
\end{align*}

The $\stepValidator$ function takes a query and a response, and computes the
validator transition by pattern matching on argument $(v:\Validator)$.  Let
$\beta$ be the validator state type of $v$, $\vstep$ be the loop body and
$(state:\beta)$ the current state of $v$.  Then $\stepValidator(q,a)(v)$ calls
the $\vstep$ function to validate the $Q\times A$ pair.  If the pair is valid,
then $\vstep$ returns a post-validation $state'$, which replaces the validator's
current $state$.  Otherwise, the validator halts with $\None$.
\end{definition}

\begin{definition}[Trace acceptance in QAC]
Validator $v$ accepts trace $t$ if and only if it {\em cosumes} the trace and
steps to some validator $v'$, written as ``$\behaves vtv'~$'':
\[\accepts v t\quad\triangleq\quad\exists v',\behaves v t v'\]
Here the consumability relation is defined as follows:
\begin{enumerate}
\item A validator can consume an empty trace and step to itself:
  \[\behaves v\nil v\]
\item A validator consumes a non-empty trace if it can consume the head of the
  trace, and step to some validator that consumes the tail of the trace:
  \[\behaves v {t+(q,a)} v_2\quad\triangleq\quad\exists v_1,\behaves v t v_1\wedge
  \behaves{v_1}{(q,a)}{v_2}\]
\end{enumerate}
\end{definition}
