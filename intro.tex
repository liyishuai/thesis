We trust programs by testing them, but how do we trust our testers?

Software engineering requires rigorous testing of rapidly evolving programs,
which costs manpower comparable to developing the product itself.  To guarantee
programs' compliance with the specification, we need testers that can tell
compliant implementations from violating ones.

\section{Interactive Testing}
Suppose we want to test a web server that supports GET and PUT methods:
\begin{lstlisting}[style=customcoq]
  CoFixpoint server (data: key -> value) :=
    request <- recv;;
    match request with
    | GET k   => send (data k);; server  data
    | PUT k v => send  Done   ;; server (data [k |-> v])
    end.
\end{lstlisting}
We can write a tester client that interacts with the server and determines
whether it behaves correctly:
\begin{lstlisting}[style=customcoq]
  CoFixpoint tester (data: key -> value) :=
    request <- random;;
    send request;;
    response <- recv;;
    match request with
    | GET k   => if response =? data k
                 then tester data
                 else reject
    | PUT k v => if response =? Done
                 then tester (data [k |-> v])
                 else reject
    end.
\end{lstlisting}
This tester implements a reference server internally that computes the expected
behavior.  The behavior is then compared against that produced by the system
under test (SUT).  The tester rejects the SUT upon any difference from the
computed expectation.

Such method only works for deterministic systems whose behavior can be precisely
computed from its input.  Whereas, many systems are allowed to behave
nondeterministically.

For example, \http servers may generate entity tags (ETags) to represent its
resources' versions~\cite{rfc7232}.  The server may implement ETags with
arbitrary algorithms {\it e.g.} hashing, timestamps {\it etc.}, making its value
unpredictable from the clients' requests.  As a result, the tester should not
compare the server's behavior against some expectation, but evaluate whether its
behavior is producible by some valid implementation.

\section{Nondeterminism}

When testing server implementations against protocol specifications, one
critical challenge is {\em nondeterminism}, which arises in two forms---we call
them (1) {\em internal nondeterminism} and (2) {\em network nondeterminism}:

(1) {\em Within} the server, correct behavior may be \mbox{underspecified}.
For example, to handle HTTP conditional requests \cite{rfc7232}, a server
generates strings called entity tags (ETags), but the RFC specification does
not limit {what} values these ETags should be.  Thus, to create test
messages containing ETags, the tester must remember and reuse the ETags it
has been given in previous messages from the server.

(2) {\em Beyond} the server, messages and responses between the server and
different clients might be delayed and reordered by the network and
operating-system buffering.  If the tester cannot control how the execution
environment reorders messages---{\it e.g.,} when testing over the Internet---it
needs to specify what servers are valid as observed over the network.

These sources of nondeterminism pose challenges in various aspects of testing network
protocols: (i) The {\em validation logic} should accept various implementations,
as long as the behavior is included in the specification's space of
uncertainties; (ii) To capture bugs effectively, the {\em test harness} should
generate test cases based on runtime observations; (iii) When {\em shrinking} a
counterexample, the test harness should adjust the test cases based on the
server's behavior, which might vary from one execution to another.

To address these challenges, I introduce symbolic languages for writing
specifications and representing test cases:

(i) The specification is written as a reference implementation---a
nondeterministic program that exhibits all possible behavior allowed by
the protocol.  Inter-implementation and inter-execution uncertainties are
represented by symbolic variables, and the space of nondeterministic behavior is
defined by all possible assignments of the variables.

The validation logic is derived from the reference implementation, by {\em
  dualising} the server-side program into a client-side observer.

(ii) Test generation heuristics are defined as computations from the observed
trace (list of sent and received messages) to the next message to send.  I
introduce a symbolic intermediate representation for specifying the relation
between the next message and previous messages.

(iii) The symbolic language for generating test cases also enables effective
shrinking of test cases.  The test harness minimizes the counterexample by
shrinking its symbolic representation.  When running the test with a shrunk
input, the symbolic representations can be re-instantiated into request messages
that reflect the original heuristics.

\section{Contribution}
Symbolic abstract representation can address challenges in testing networked systems with uncertain behavior.
Specifying protocols with symbolic reference implementation enables validating
the system's behavior systematically.  Representing test input as abstract
messages allows generating and shrinking interesting test cases.  Combining
these methods result in a rigorous tester that can capture protocol violations
effectively.

\section{Outline}
This thesis is structured as follows:
