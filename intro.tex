We trust programs by testing them, but how do we trust our testers?

Software engineering requires rigorous testing of rapidly evolving programs,
which costs manpower comparable to developing the product itself.  To guarantee
programs' compliance with the specification, we need testers that can tell
compliant implementations from violating ones.

\section{Interactive Testing}
Suppose we want to test a web server that supports GET and PUT methods:
\begin{lstlisting}[style=customcoq]
  CoFixpoint server (data: key -> value) :=
    request <- recv;;
    match request with
    | GET k   => send (data k);; server  data
    | PUT k v => send  Done   ;; server (data [k |-> v])
    end.
\end{lstlisting}
We can write a tester client that interacts with the server and determines
whether it behaves correctly:
\begin{lstlisting}[style=customcoq]
  CoFixpoint tester (data: key -> value) :=
    request <- random;;
    send request;;
    response <- recv;;
    match request with
    | GET k   => if response =? data k
                 then tester data
                 else reject
    | PUT k v => if response =? Done
                 then tester (data [k |-> v])
                 else reject
    end.
\end{lstlisting}
This tester implements a reference server internally that computes the expected
behavior.  The behavior is then compared against that produced by the system
under test (SUT).  The tester rejects the SUT upon any difference from the
computed expectation.

Such method only works for deterministic systems whose behavior can be precisely
computed from its input.  Whereas, many systems are allowed to behave
nondeterministically.

For example, \http servers may generate entity tags (ETags) to represent its
resources' versions~\cite{rfc7232}.  The server may implement ETags with
arbitrary algorithms {\it e.g.} hashing, timestamps {\it etc.}, making its value
unpredictable from the clients' requests.  As a result, the tester should not
compare the server's behavior against some expectation, but evaluate whether its
behavior is producible by some valid implementation.

\section{Nondeterminism}

When testing server implementations against protocol specifications, one
critical challenge is {\em nondeterminism}, which arises in two forms---we call
them (1) {\em internal nondeterminism} and (2) {\em network nondeterminism}:

(1) {\em Within} the server, correct behavior may be \mbox{underspecified}.
For example, to handle HTTP conditional requests \cite{rfc7232}, a server
generates strings called entity tags (ETags), but the RFC specification does
not limit {what} values these ETags should be.  Thus, to create test
messages containing ETags, the tester must remember and reuse the ETags it
has been given in previous messages from the server.

(2) {\em Beyond} the server, messages and responses between the server and
different clients might be delayed and reordered by the network and
operating-system buffering.  If the tester cannot control how the execution
environment reorders messages---{\it e.g.,} when testing over the Internet---it
needs to specify what servers are valid as observed over the network.

These sources of nondeterminism pose challenges in various aspects of testing network
protocols: (i) The {\em validation logic} should accept various implementations,
as long as the behavior is included in the specification's space of
uncertainties; (ii) To capture bugs effectively, the {\em test harness} should
generate test cases based on runtime observations; (iii) When {\em shrinking} a
counterexample, the test harness should adjust the test cases based on the
server's behavior, which might vary from one execution to another.

To address these challenges, I introduce symbolic languages for writing
specifications and representing test cases:

(i) The specification is written as a reference implementation---a
nondeterministic program that exhibits all possible behavior allowed by
the protocol.  Inter-implementation and inter-execution uncertainties are
represented by symbolic variables, and the space of nondeterministic behavior is
defined by all possible assignments of the variables.

The validation logic is derived from the reference implementation, by {\em
  dualising} the server-side program into a client-side observer.

(ii) Test generation heuristics are defined as computations from the observed
trace (list of sent and received messages) to the next message to send.  I
introduce a symbolic intermediate representation for specifying the relation
between the next message and previous messages.

(iii) The symbolic language for generating test cases also enables effective
shrinking of test cases.  The test harness minimizes the counterexample by
shrinking its symbolic representation.  When running the test with a shrunk
input, the symbolic representations can be re-instantiated into request messages
that reflect the original heuristics.

\section{Contribution}
Symbolic abstract representation can address challenges in testing networked systems with uncertain behavior.
Specifying protocols with symbolic reference implementation enables validating
the system's behavior systematically.  Representing test input as abstract
messages allows generating and shrinking interesting test cases.  Combining
these methods result in a rigorous tester that can capture protocol violations
effectively.

\section{Outline}
This thesis is structured as follows:

\section{Introduce motivation in this section entirely}
The Deep Specification project~\cite{deepspec} aims at building a web server and
guarantee its functional correctness with respect to formal specification of the
network protocol.

\http requests can be conditional: if the client has a local copy of some
resource and the copy on the server has not changed, then the server needn't
resend the resource.  To achieve this, an \http server may generate a short
string, called an ``entity tag'' (ETag), identifying the content of some
resource, and send it to the client:


\begin{center}
  \begin{minipage}[t]{.4\textwidth}
    \begin{lstlisting}[style=customc]
/* Client: */
GET /target HTTP/1.1
    \end{lstlisting}
  \end{minipage}\begin{minipage}[t]{.4\textwidth}
    \begin{lstlisting}[style=customc]
/* Server: */
HTTP/1.1 200 OK
ETag: "tag-foo"
... content of /target ...
    \end{lstlisting}
  \end{minipage}
\end{center}

The next time the client requests the same resource, it can include the ETag in
the GET request, informing the server not to send the content if its ETag still
matches:
\begin{center}
\begin{minipage}[t]{.4\textwidth}
\begin{lstlisting}[style=customc]
/* Client: */
GET /target HTTP/1.1
If-None-Match: "tag-foo"
\end{lstlisting}
\end{minipage}\begin{minipage}[t]{.4\textwidth}
\begin{lstlisting}[style=customc]
/* Server: */
HTTP/1.1 304 Not Modified
\end{lstlisting}
\end{minipage}
\end{center}
If the ETag does not match, the server responds with code 200 and the updated
content as usual.

Similarly, if a client wants to modify the server's resource atomically by
compare-and-swap, it can include the ETag in the PUT request as
\inlinec{If-Match} precondition, which instructs the server to only update the
content if its current ETag matches:
\begin{center}
  \begin{minipage}[t]{.4\textwidth}
    \begin{lstlisting}[style=customc]
/* Client: */
PUT /target HTTP/1.1
If-Match: "tag-foo"
... content (A) ...
    \end{lstlisting}
  \end{minipage}
  \begin{minipage}[t]{.4\textwidth}
    \begin{lstlisting}[style=customc]
/* Server: */
HTTP/1.1 204 No Content
    \end{lstlisting}
  \end{minipage}

  \begin{minipage}[t]{.4\textwidth}
    \begin{lstlisting}[style=customc]
/* Client: */
GET /target HTTP/1.1
    \end{lstlisting}
  \end{minipage}
  \begin{minipage}[t]{.4\textwidth}
    \begin{lstlisting}[style=customc]
/* Server: */
HTTP/1.1 200 OK
ETag: "tag-bar"
... content (A) ...
    \end{lstlisting}
  \end{minipage}
\end{center}
If the ETag does not match, then the server should not perform the requested
operation, and should reject with code 412:
\begin{center}
  \begin{minipage}[t]{.4\textwidth}
    \begin{lstlisting}[style=customc]
/* Client: */
PUT /target HTTP/1.1
If-Match: "tag-baz"
... content (B) ...
    \end{lstlisting}
  \end{minipage}
  \begin{minipage}[t]{.4\textwidth}
    \begin{lstlisting}[style=customc]
/* Server: */
HTTP/1.1 412 Precondition Failed
    \end{lstlisting}
  \end{minipage}

  \begin{minipage}[t]{.4\textwidth}
    \begin{lstlisting}[style=customc]
/* Client: */
GET /target HTTP/1.1
    \end{lstlisting}
  \end{minipage}
  \begin{minipage}[t]{.4\textwidth}
    \begin{lstlisting}[style=customc]
/* Server: */
HTTP/1.1 200 ok
ETag: "tag-bar"
... content (A) ...
    \end{lstlisting}
  \end{minipage}
\end{center}

If the tag does not match, the server responds with code 200 and the updated
content as usual.  Similarly, if a client wants to modify the server's resource
atomically by compare-and-swap, it can include the ETag in the PUT request as
\inlinec{If-Match} precondition, which instructs the server to only update the
content if its current ETag matches.

Thus, whether a server's response should be judged {\em valid} or not
depends on the ETag it generated
when creating the resource.  If the tester doesn't know the server's internal
state ({\it e.g.}, before receiving any 200 response including the ETag), and
cannot enumerate all of them (as ETags can be arbitrary strings), then it needs
to maintain a space of all possible values, narrowing the space upon further
interactions with the server.

\begin{figure}
  \begin{lstlisting}[style=customcoq,mathescape=true]
(* update : (K -> V) * K * V -> (K -> V) *)
let check (trace  : stream http_message,
           data   : key -> value,
           is     : key -> etag,
           is_not : key -> list etag) =
  match trace with
  | PUT(k,t,v) :: SUCCESSFUL :: tr' =>
    if t $\in$ is_not[k] then reject
    else if   is[k] == unknown
            $\vee$ strong_match(is[k],t)
         then let d' = update(data,k,v)     in
              let i' = update(is,k,unknown) in
              let n' = update(is_not,k,[])  in
       (* Now the tester knows that
        * the data in [k] is updated to [v],
        * but its new ETag is unknown. *)
              check(tr',d',i',n')
         else reject
  | PUT(k,t,v) :: PRECONDITION_FAILED :: tr' =>
    if strong_match(is[k],t) then reject
    else let n' = update(is_not, k, t::is_not[k])
      (* Now the tester knows that
       * the ETag of [k] is other than [t]. *)
         in check(tr',data,is,n')
  | GET(k,t) :: NOT_MODIFIED :: tr' =>
    if t $\in$ is_not[k] then reject
    else if is[k] == unknown $\vee$ weak_match(is[k],t)
         then let i' = update(is,k,t) in
       (* Now the tester knows that
        * the ETag of [k] is equal to [t]. *)
              check(tr',data,i',is_not)
         else reject
  | GET(k,t0) :: OK(t,v) :: tr' =>
    if weak_match(is[k],t0) then reject
    else if data[k] $\neq$ unknown $\wedge$ data[k] $\neq$ v
         then reject
         else let d' = update(data,k,v) in
              let i' = update(is,  k,t) in
       (* Now the tester knows
        * the data and ETag of [k]. *)
              check(tr',d',i',is_not)
  | _ :: _ :: _  => reject
  end
  \end{lstlisting}
  \caption{Ad hoc tester for \http conditional requests, demonstrating how
    tricky it is to write the logic by hand.  The checker determines whether a
    one-client-at-a-time \ilc{trace} is valid or not.  The trace is represented
    as a stream (infinite linked list, constructed by ``\ilc{::}'') of HTTP
    messages sent and received.
    \ilc{PUT(k,t,v)} represents a PUT
    request that changes \ilc{k}'s value into \ilc{v} only if its ETag matches
    \ilc{t}; \ilc{GET(k,t)} is a GET request for \ilc{k}'s value only if its
    ETag does not match \ilc{t}; \ilc{OK(t,v)} indicates the request target's
    value is \ilc{v} and its ETag is \ilc{t}.  The tester maintains three
    sorts of  knowledge about
    the server: \ilc{data} stored for each content, what some
    ETag \ilc{is} known to be equal to, and what some ETag \ilc{is_not} equal
    to.
  }
  \label{fig:etag-tester}
\end{figure}

It is possible, but tricky, to write an ad hoc tester for \http by manually
``dualizing'' the behaviors described by the informal specification documents
(RFCs).  The protocol document describes {\em how} a valid server should handle
requests, while the tester needs to determine {\em what} responses received from
the server are valid.  For example, ``If the server has revealed some resource's
ETag as \inlinec{"foo"}, then it must not reject requests targetting this
resource conditioned over \inlinec{If-Match: "foo"}, until the resource has been
modified''; and ``Had the server previously rejected an \inlinec{If-Match}
request, it must reject the same request until its target has been modified.''
\autoref{fig:etag-tester} shows a hand-written tester for checking this bit of
ETag functionality; we hope the reader will agree that this testing logic is not
straightforward to derive from the informal ``server's eye'' specifications.

Networked systems are naturally concurrent, as a server can be connected with
multiple clients.  The network might delay packets indefinitely, so messages
sent via different channels may be reordered during transmission.  When the
tester observes messages sent and received on the client side, it should allow
all observations that can be explained by the combination of a valid server + a
reasonable network environment between the server and clients.

