Testers are programs that determine whether implementations are compliant or
not, based on its observations.  This section defines the basic concepts and
notations in interactive testing.

\begin{definition}[Implementations and Traces]
  {\em Implementations} are programs that can interact with their environment.
  {\em Traces} are the implementations' inputs and outputs during execution.
  ``Implementation $i$ can {\em produce} trace $t$'' is written as ``$\behaves i
  t$''.
\end{definition}

This chapter focuses on synchronous testing and assumes no external
nondeterminism.  Here the tester's observation is identical to the
implementation's output, so the tester-side trace is the same as that on the
implementation side.  Asynchronous testing will be discussed in
\autoref{chap:practices}.

\begin{definition}[Specification, Validity, and Compliance]
  \label{def:compliance}
  A {\em specification} is a description of valid traces.  ``Trace $t$ is {\em
    valid} per specification $s$'' is written as ``$\valid s t$''.

  An implementation $i$ {\em complies} with a specification $s$ (written as
  ``$\complies s i$'') if it only produces traces that are valid per the
  specification:
  \[\complies s i\quad\triangleq\quad\forall t,(\behaves i t)\implies\valid s t\]
\end{definition}

\begin{definition}[Tester components and correctness]
  \label{def:tester}
  A tester consists of (i) a {\em validator} that accepts or rejects
  traces (written as ``$\accepts v t$'' and ``$\rejects v t$''), and
  (ii) a {\em test harness} that triggers different traces with
  various inputs.

  A tester is {\em correct} if its acceptances and rejections are sound and
  complete.  A tester is {\em rejection-sound} if it rejects only non-compliant
  implementations; it is {\em rejection-complete} if it can reject all
  non-compliant implementations, provided sufficient time of
  execution.\footnote{The semantics of ``soundness'' and ``completeness'' vary
    among contexts.  This thesis inherits terminologies from existing
    literature~\cite{Tretmans}, but explicitly uses ``rejection-'' prefix for
    clarity.  ``Rejection soundness'' is equivalent to ``acceptance
    completeness'', and vice versa.}
\end{definition}

The tester's correctness is based on its components' properties: A
rejection-sound tester requires its validator to be rejection-sound; A
rejection-complete tester consists of (i) a rejection-complete validator and
(ii) an exhaustive test harness that can eventually trigger invalid traces.  The
validators' soundness and completeness are defined as follows:

\begin{definition}[Correctness of validators]
  A validator $v$ is {\em rejection-sound} with respect to specification $s$
  (written as ``$\rejSound v s$'') if it only rejects traces that are invalid
  per $s$:
  \[\rejSound v s\quad\triangleq\quad\forall t,\rejects v t\implies\invalid s t\]

  A validator $v$ is {\em rejection-complete} with respect to specification $s$
  (written as ``$\rejComplete v s$'') if it rejects all behaviors that are
  invalid per $s$:
  \[\rejComplete v s\quad\triangleq\quad\forall t,\invalid s t\implies\rejects v t\]
\end{definition}

The rest of this chapter shows how to construct specifications and validators,
and how to prove the validators' correctness with respect to the specifications.
