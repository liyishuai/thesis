During the testing practice in \autoref{chap:practices}, the tester's quality was
evaluated by mutation testing, {\it i.e.} running the tester against buggy
implementations to see if it rejects.  To formally prove that the tester is
good, I develop a theory for reasoning on testers' good properties.

{\em Interactive testing} is a process that reveals the SUT's interactions and
determines whether it satisfies the specification.  There are two kinds of
interactions: (1) {\em inputs} that the tester can specify, and (2) {\em
  outputs} that are observed from the SUT.  In particular, when testing
networked systems, the input is a message sent by the tester, and the output is
a message received from the SUT.

When viewing the SUT as a function from inputs to outputs, we can test the
system by (1) providing an input, (2) get the output, and (3) validating the
input-output pair.  This process is called {\em synchronous testing}.

However, the nature of networked systems is that multiple messages might arrive
at the system simultaneously, and a high-throughput system should handle the
messages concurrently.  To check the system's validity upon concurrent inputs,
the tester should send multiple messages, rather than executing ``one client at
a time''.  This non-blocking process is called {\em asynchronous testing}.

My goal is to formalise the techniques in \textcite{issta21} into a generic
theory for asynchronous testing.

A tester consists of two parts: (i) a test harness that interacts with the SUT
and observes the interactions, and (ii) a validator that determines whether the
observations satisfy the specification.

The test harness needs to produce counterexamples effectively, and provide good
coverage of test cases.  The goal is to locate unknown bugs within a fixed
budget, which is more practical than theoretical, and will be discussed in
\autoref{sec:harness}.  The test theory in this dissertation focuses on
guaranteeing the soundness and completeness of the validator logic.
