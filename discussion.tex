This dissertation presented a systematic technique for testing interactive systems
with uncertain behavior.  I proposed a theory of dualizing protocol
specification into validators, with formal guarantee of soundness and
completeness (\autoref{chap:theory}).  To test systems in real-world practice, I
applied the dualization theory to the interaction tree specification language,
and derived specifications into interactive testing programs
(\autoref{chap:practices}).  I then presented a test harness design to generate
and shrink interactive test inputs effectively (\autoref{chap:harness}).  The
entire methodology was evaluated by detecting programs' incompliance with the
specification using automatically derived testers (\autoref{chap:eval}).

To address challenges posed by internal, external, and inter-execution
nondeterminism, I introduced various flavors of symbolic abstract
interpretation.  Systems' internal choices are represented as symbolic variables
and unified against the tester's observations (\autoref{sec:dualize-prog},
\autoref{sec:internal-nondet}).  Possible impacts made by the environment are
represented as nondeterministic branches (\autoref{sec:external-nondet}) and
determined by backtrack searching (\autoref{sec:backtrack}).  The test inputs
are generated as symbolic intermediate representations that can adapt to
different traces during runtime (\autoref{sec:shrinking}).

The techniques in this thesis can be expanded in several dimensions:
\paragraph{Specifying and testing in various scenarios}
The Unison file synchronizer in \autoref{sec:sync} was specified and tested as a
command line tool that is manually triggered.  Whereas, Unison can run as a
background process that monitors the file system and automatically propagates
the updates, just like Dropbox.  Running in the background introduces more
nondeterminism, e.g., whether the tester's modification has been buffered or
propagated to the file system, whether the operating system has scheduled the
synchronizer process to check the file system's updates, whether the
synchronizer has finished propagating the changes to the other replica, {\it
etc}.  Migrating testing scenarios might expose the limitations of my
methodology or introduce new challenges in testing.

\paragraph{Integrating other testing techniques to the framework}
The experiments with Apache and Unison have shown that some bugs are
time-related, e.g., only revealed by specific request and response orders, or
require conflicting modifications to occur within the same millisecond.  My
current test harness doesn't tune the timing of executions, and might benefit
from integrating packet dynamics by \citet{pkt-dyn}.

The randomized generator in my framework produced a large number of inputs with
guidance of heuristics based on domain-specific knowledges.  To find bugs with
fewer tests, \citet{judge-cover} applied ideas from combinatorial testing to
tune the generators' distributions for better coverage.  Measuring and improving
coverage of asynchronous tests is yet to be studied.
