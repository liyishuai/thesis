This thesis presents a systematic technique for testing interactive systems with
uncertain behavior.  I propose a theory of dualizing protocol specification into
validators, with formal guarantee of soundness and completeness
(\autoref{chap:theory}).  To test systems in real-world practices, I applied the
dualization theory to the interaction tree specification language, and derived
specifications into interactive testing programs (\autoref{chap:practices}).  I
then presented a test harness design to generate and shrink interactive test
inputs effectively (\autoref{chap:harness}).  The entire methodology is
evaluated by revealing programs' incompliant behavior with derived testers
(\autoref{chap:eval}).

To address challenges posed by internal, external, and inter-execution
nondeterminisms, I introduced various flavors of symbolic abstract
interpretation: Systems' internal choices are represented as symbolic variables
and unified against the tester's observations (\autoref{sec:dualization},
\autoref{sec:internal-nondet}); Possible impacts made by the environment are
represented as nondeterministic branches (\autoref{sec:external-nondet}), and
determined by backtrack searching (\autoref{sec:backtrack}); The test inputs are
generated as symbolic intermediate representations that can adapt to different
traces during runtime (\autoref{sec:shrinking}).

The technique in this thesis can be expanded in scenario and by combination:
\begin{enumerate}
\item Specifying and testing other kinds of SUTs in various setups {\it e.g.}
  background monitoring processes, distributed locks, cyber-physical systems
  {\it etc}.  These attempts might expose the limitations of my methodology or
  introduce new challenges in testing.

\item Integrating other testing techniques to the framework {\it e.g.} tuning
  the executor to capture time-related violations~\cite{pkt-dyn}, find bugs with
  fewer tests by enumerating within a small input space~\cite{judge-cover}, {\it
    etc}.  This may enhance my testers' effectiveness or reveal potential
  improvements that existing techniques can make.
\end{enumerate}
