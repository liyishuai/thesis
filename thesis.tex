%% Antal Spector-Zabusky's UPenn thesis formatting skeleton, 2021
%%
%% Feel free to use any of this as necessary; you can contact me about it at
%% <antal.b.sz@gmail.com>.  This isn't all the styling in my thesis; it's mostly
%% limited to things that help you satisfy the formatting guidelines instead of
%% style choices.  It does not contain styling for things I didn't include, such
%% as a dedication or an index.  I wrote this for a CS thesis, so some details
%% may be specific to that.
%%
%% This should follow the formatting guidelines from
%% <https://guides.library.upenn.edu/dissertation_manual/formatting> as of
%% Spring 2021.
%%
%% Explanatory comments are annotated with ASZ; you can delete them if you
%% want.
%%
%% Good luck!

%% ASZ: 12 point font, single-sided (don't alternate left/right margins), and
%% sensible defaults.
\documentclass[12pt,oneside]{amsbook}

%%%%%%%%%%%%%%%%%%%%%%%%%%%%%%%%%%%%%%%%%%%%%%%%%%%%%%%%%%%%%%%%%%%%%%%%%%%%%%%%

%% ASZ: My usual "make things look nice and behave well" block
\usepackage[utf8]{inputenc}
\usepackage[T1]{fontenc}
\usepackage{lmodern}
\usepackage{microtype}
\usepackage{accsupp}
\usepackage{etoolbox}
\usepackage{newunicodechar}
\usepackage[dvipsnames,svgnames,x11names]{xcolor}
\usepackage{hyperref}
\usepackage{xspace}
\usepackage[
  backend=biber]{biblatex}
\addbibresource{bibliography.bib}

\usepackage{draft} % local
\newif\ifdraft\drafttrue
\newnote{lys}{BrickRed}
\newnote{bcp}{Chocolate}
\newnote{sz}{violet}

\newcommand{\http}{HTTP/1.1\xspace}

%% ASZ: `amsbook` didn't style `\paragraph`s appropriately
\patchcmd{\paragraph}{\normalfont}{\itshape}
  {} % Success
  {\GenericError
     {} % Continuation for \MessageBreak (unused)
     {Could not redefine \string\paragraph\space to be italic}
     {(This is a hack in the preamble.)} % Where to look for more info?
     {I tried to replace \string\normalfont\space with \string\itshape, but I
      couldn't.}}

%% ASZ: Change the spacing in the list of figures – I don't know if you need
%% this or not
\makeatletter
\patchcmd{\l@figure}{1.5pc}{2.25pc}
  {} % Success
  {\GenericError
     {} % Continuation for \MessageBreak (unused)
     {Could not fix the spacing in the list of figures}
     {(This is a hack in the preamble.)} % Where to look for more info?
     {I tried to replace the 1.5pc figure number--figure description separation
      with a 2.25pc separation in \string\l@figure\space, but I couldn't.}}

%% ASZ: I like it when tables of contents know where they are
\patchcmd{\@starttoc}{\ifx\contentsname}{\iffalse}
  {} % Success
  {\GenericError
     {} % Continuation for \MessageBreak (unused)
     {Could not add a line to the table of contents for itself}
     {(This is a hack in the preamble.)} % Where to look for more info?
     {I tried to replace the check in \string\@starttoc\space that suppresses a
      table of contents entry for the table of contents, but I couldn't.}}

%% ASZ: The other half of the above
\patchcmd{\@starttoc}{\@tocwrite}{\phantomsection\@tocwrite}
  {} % Success
  {\GenericError
     {} % Continuation for \MessageBreak (unused)
     {Could not add link targets to the table of contents/front matter lists}
     {(This is a hack in the preamble.)} % Where to look for more info?
     {I tried to add \string\phantomsection\space to \string\@starttoc\space so
      that it would generate hyperlink targets, but I couldn't.}}
\makeatother

%% ASZ: Set the page layout correctly: 1.5" margins on the left, 1" margins
%% everywhere else, page numbers count
\usepackage{fullpage}
\usepackage[margin=1in,left=1.5in,includeheadfoot]{geometry}

%% ASZ: Make footnote numbers run continuously, rather than restart at the start
%% of every chapter; this is required by the formatting guidelines
\counterwithout*{footnote}{chapter}

%% ASZ: I found these to be much clearer
\numberwithin{section}{chapter}
\numberwithin{figure}{chapter}
\numberwithin{equation}{chapter}

%%%%%%%%%%%%%%%%%%%%%%%%%%%%%%%%%%%%%%%%%%%%%%%%%%%%%%%%%%%%%%%%%%%%%%%%%%%%%%%%

%% ASZ: This is a package for generating and styling the various "preliminary
%% pages": title page, copyright page, and abstract
\usepackage{upenn-dissertation-preliminary-pages} % local
\NoSignatureLines %% ASZ: For COVID – turns off adding lines on the title page
                  %% for people to sign.  The default is \YesSignatureLines.
\SingleSpaceAbstract %% ASZ: I think it looks better than a double-spaced
                     %% abstract, although in deference to the guidelines,
                     %% \DoubleSpaceAbstract is the default.

%% ASZ: Fill out your own information here :-)  Some notes:
%%
%% * You need to specify your advisor, the graduate chair, and your committee
%%   members, even if there's overlap between these groups.
%%
%%   * Your external committee member(s) need both their title and their
%%   affiliation; below, just imagine that everyone but Gödel works at Penn.
%%
%% * Note that some of the faculty involved for you may be assistant or
%%   associate professors (but I couldn't bring myself to apply that title to
%%   any of these luminaries).  I don't believe extra titles (e.g., "ENIAC
%%   President's Distinguished Professor") are necessary.
%%
%% * Only one committee chair is supported; you'll need to modify
%%   `upenn-dissertation-preliminary-pages.sty` (or email me about it) to change
%%   that.
%%
%% * When rendering the title page, your chair (if present) comes first,
%%   followed by the rest of the committee in the same order as in the TeX.
%%
%% * Using a Creative Commons license is optional, and thus so is specifying
%%   `\creativecommons`.  If you want to use one, choose from the licenses
%%   available at <https://creativecommons.org/licenses/>, perhaps by using the
%%   license chooser at <https://creativecommons.org/choose/>.  You'll have to
%%   copy the name of the  license (including the short string) and the URL
%%   separately, as you can see with the text below (the license I used); I
%%   didn't build a LaTeX Creative Commons license parser (yet? :-)).
\title{Language-Based Interactive Testing}
\author{Yishuai Li}
\graduategroup{Computer and Information Science}
\graduationyear{2022}
\advisor{Benjamin C. Pierce}{Professor of Computer and Information Science}
\graduatechair{Mayur Naik}{Professor of Computer and Information Science}
\committeechair{Steve Zdancewic}{Professor of Computer and Information Science}
\committeemember{Mayur Naik}{Professor of Computer and Information Science}
\committeemember{Boon Thau Loo}{Professor of Computer and Information Science}
\committeemember{John Hughes}{Professor of Computing Science, Chalmers University of Technology}
\creativecommons
  {Attribution-Share\-Alike 4.0 International (CC BY-SA 4.0)}
  {https://creativecommons.org/licenses/by-sa/4.0/}

\begin{document}

%% ASZ: Roman page numbers
\frontmatter

\maketitle

\copyrightpage

\chapter*{Acknowledgments}
\label{chap:acknowledgments}

Lorem ipsum dolor sit amet, consectetur adipiscing elit, sed do eiusmod tempor
incididunt ut labore et dolore magna aliqua. Ut enim ad minim veniam, quis
nostrud exercitation ullamco laboris nisi ut aliquip ex ea commodo
consequat. Duis aute irure dolor in reprehenderit in voluptate velit esse cillum
dolore eu fugiat nulla pariatur. Excepteur sint occaecat cupidatat non proident,
sunt in culpa qui officia deserunt mollit anim id est laborum.

%% ASZ: ... the rest of your acknowledgments go here ...

%% ASZ: The `upenn-dissertation-preliminary-pages` package will correctly style
%% this, including your thesis name, your name, and your advisor's name.
\begin{abstract}{chap:abstract}
  Lorem ipsum dolor sit amet, consectetur adipiscing elit, sed do eiusmod tempor
  incididunt ut labore et dolore magna aliqua. Ut enim ad minim veniam, quis
  nostrud exercitation ullamco laboris nisi ut aliquip ex ea commodo
  consequat. Duis aute irure dolor in reprehenderit in voluptate velit esse
  cillum dolore eu fugiat nulla pariatur. Excepteur sint occaecat cupidatat non
  proident, sunt in culpa qui officia deserunt mollit anim id est laborum.
\end{abstract}

\tableofcontents
\listoffigures

%% ASZ: Arabic page numbers
\mainmatter

\chapter{Introduction}
\label{chap:introduction}

The security and robustness of networked systems rest in large part on the
correct behavior of various sorts of servers.  This can be validated either by
full-blown verification or model checking against formal specifications, or
less expensively by rigorous testing.

Rigorous testing requires a rigorous specification of the protocol that we
expect the server to obey.  Protocol specifications can be written as (i) a {\em
  server model} that describes {\em how} valid servers should handle messages,
or (ii) a {\em property} that defines {\em what} server behaviors are valid.
From these specifications, we can conduct (i) {\em model-based
  testing}~\cite{broy2005model} or (ii) {\em property-based testing}~\cite{pbt},
respectively.

When testing server implementations against protocol specifications, one
critical challenge is {\em nondeterminism}, which arises in two forms---we call
them (1) {\em internal nondeterminism} and (2) {\em network nondeterminism}:

(1) {\em Within} the server, correct behavior may be \mbox{underspecified}.
For example, to handle HTTP conditional requests \cite{rfc7232}, a server
generates strings called entity tags (ETags), but the RFC specification does
not limit {what} values these ETags should be.  Thus, to create test
messages containing ETags, the tester must remember and reuse the ETags it
has been given in previous messages from the server.

(2) {\em Beyond} the server, messages and responses between the server and
different clients might be delayed and reordered by the network and
operating-system buffering.  If the tester cannot control how the execution
environment reorders messages---{\it e.g.,} when testing over the Internet---it
needs to specify what servers are valid as observed over the network.

These sources of nondeterminism pose challenges in various aspects of testing network
protocols: (i) The {\em validation logic} should accept various implementations,
as long as the behavior is included in the specification's space of
uncertainties; (ii) To capture bugs effectively, the {\em test harness} should
generate test cases based on runtime observations; (iii) When {\em shrinking} a
counterexample, the test harness should adjust the test cases based on the
server's behavior, which might vary from one execution to another.

To address these challenges, I introduce symbolic languages for writing
specifications and representing test cases:

(i) The specification is written as a reference implementation---a
nondeterministic program that exhibits all possible behavior allowed by
the protocol.  Inter-implementation and inter-execution uncertainties are
represented by symbolic variables, and the space of nondeterministic behavior is
defined by all possible assignments of the variables.

The validation logic is derived from the reference implementation, by {\em
  dualising} the server-side program into a client-side observer.

(ii) Test generation heuristics are defined as computations from the observed
trace (list of sent and received messages) to the next message to send.  I
introduce a symbolic intermediate representation for specifying the relation
between the next message and previous messages.

(iii) The symbolic language for generating test cases also enables effective
shrinking of test cases.  The test harness minimizes the counterexample by
shrinking its symbolic representation.  When running the test with a shrunk
input, the symbolic representations can be re-instantiated into request messages
that reflect the original heuristics.

\paragraph{Thesis claim}
Symbolic abstract representation can address challenges in testing networked systems with uncertain behavior.
Specifying protocols with symbolic reference implementation enables validating
the system's behavior systematically.  Representing test input as abstract
messages allows generating and shrinking interesting test cases.  Combining
these methods result in a rigorous tester that can capture protocol violations
effectively.

This thesis is structured as follows:

%% ASZ: ... the rest of your thesis goes here ...

\printbibliography
  %% ASZ: AucTeX (the Emacs package for LaTeX I used) doesn't support a `.bib`
  %% file and a `.tex` file with the same name

\end{document}
